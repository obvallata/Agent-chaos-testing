\newpage
\begin{center}
  \textbf{\large ПРОДЕЛАННАЯ РАБОТА ПО НИР НА ТЕМУ СИСТЕМА АВТОМАТИЧЕСКОГО ХАОС-ТЕСТИРОВАНИЯ ХОСТОВОГО АГЕНТА}
\end{center}
\refstepcounter{chapter}
\addcontentsline{toc}{chapter}{ПРОДЕЛАННАЯ РАБОТА ПО НИР НА ТЕМУ СИСТЕМА АВТОМАТИЧЕСКОГО ХАОС-ТЕСТИРОВАНИЯ ХОСТОВОГО АГЕНТА}

\section{Начало}

Первым и основополагающим проделанным шагов является выбор темы НИР, научного руководителя и консультанта.
Среди потенциальных объектов исследования выступали: penetration testing для хостового агента, chaos testing для хостового агента,
статический анализ кода с точки зрения багов и потенциальных уязвимостей, система для распознавания коррелирующих событий в рамках SIEM,
безопасное выполнение удаленных adhoc запросов на хостах.
Тема хаос-тестирование хостового агента выделялась тем, что сохраняла баланс между интересной тематикой,
приемлемому объему экспертизы, который предстояло набрать и запросить, а также актуальностью с точки зрения
немедленного прикладного использования.

Научным руководителем был выбран Косицын Павел Андреевич, консультантом Кузнецов Александр.
Они являются специалистами в своих областях: Павел часто имеет дело с распределенными высоконагруженными системами
и знает много о специфичных методах тестирования; Александр занимается разработкой, в том числе хостовых агентов безопасности.

\section{Актуальность темы}

\subsection{Что такое хаос-инжиниринг?}
Хаос-инжиниринг (Chaos Engineering) — это практика в области разработки и тестирования программного обеспечения,
направленная на повышение устойчивости и надежности сложных систем (почти всегда имеются в виду распределенные системы).
Путем преднамеренного внедрения сбоев (fault injection) и непредвиденных ситуаций в контролируемой среде,
хаос-инжиниринг позволяет выявлять уязвимости и недостатки в системе до того, как они проявятся в реальной эксплуатации
и нанесут фатальный урон и компании, и пользователям.

Концепция хаос-инжиниринга стала обсуждаться в компании Netflix в конце 2000-х — начале 2010-х годов.
В процессе перехода от монолитной архитектуры к микросервисам и миграции в облако, Netflix столкнулась
с необходимостью обеспечения высокой доступности и устойчивости своих сервисов в условиях сложной распределенной системы.
Чтобы проверить и улучшить устойчивость своей инфраструктуры, инженеры Netflix разработали инструмент под названием Chaos Monkey.
Этот инструмент случайным образом отключал экземпляры служб в производственной среде, что вынуждало систему автоматически
реагировать на сбои и перенаправлять трафик, обеспечивая непрерывность обслуживания пользователей.
Успех Chaos Monkey привел к созданию целого набора инструментов, а также полноценных систем для хаос-тестирования.

\subsection{Принципы хаос-инжиниринга}

Хаос-инжиниринг основывается на нескольких ключевых принципах:
\begin{enumerate}
  \item Гипотеза о стабильном состоянии: определение нормального состояния системы на основе ключевых метрик.
  \item Внедрение сбоев: создание сценариев, максимально приближенных к реальным сбоям, которые могут произойти в системе.
  \item Минимизация радиуса воздействия: проведение экспериментов в контролируемой среде с целью ограничения потенциального
  негативного влияния на пользователей.
  \item Автоматизация экспериментов: использование инструментов для автоматического и повторяемого проведения тестов.
  \item Мониторинг внешних метрик приложения, а не его внутренних проблем:
  тщательное наблюдение за метриками системы во время экспериментов, не принимая во внимания внутреннюю стратегию деградации и отказа.
\end{enumerate}

\subsection{Инструменты хаос-инжиниринга}

Сегодня существует множество инструментов, облегчающих внедрение хаос-инжиниринга:
Gremlin, Chaos Toolkit, LitmusChaos, Chaos Mesh, AWS Fault Injection Simulator (FIS),
Azure Chaos Studio, Pumba, PowerfulSeal.
Некоторые из них разобраны в основном тексте работы с точки зрения применимости к хостовому агенту.

\subsection{Почему только распределенные системы?}

Довольно очевидна применимость хаос-инжиниринга в распределенных системах.
Современные распределенные системы характеризуются высокой сложностью из-за микросервисной архитектуры,
динамического масштабирования, распределения по зонам доступности.
В таких условиях традиционные подходы к тестированию не могут дать гарантию надежности, а хаос-инжиниринг, наоборот,
предлагает широкий диапазон сценариев для предотвращения финансовых и репутационных потерь при возникновении инцидентов.

Однако, поскольку акцент ставится именно на распределенных системах, все сценарии уходят \grqqвширь\glqq.
Популярные инструменты для хаос-инжиниринга берут во внимание огромное количество компонент инфраструктуры, в которых
может произойти отказ, что делает карту экспериментов широкой и исчерпывающий для сложно спроектированных систем.
Когда речь заходит о небольших приложениях, то есть тривиальных с точки зрения распределенных систем, такой подход
становится не столь применимым.

\subsection{Тестирование хостовых агентов}

Хостовыми агентами являются приложения, эксплуатация которых подразумевает функционирование на виртуальных или железных машинах,
к которым у разработчиков нет прямого доступа.
В современном мире хостовые агенты становятся неотъемлемой частью информационных систем, обеспечивая мониторинг,
управление и безопасность на удалённых устройствах и серверах.
Они выполняют критически важные функции, такие как сбор данных, анализ производительности, обнаружение вторжений
и реагирование на инциденты.

Но в рамках такой стратегии использования небольшие сервисы становятся не менее \grqqопасными\glqq, чем сложные системы.
Поскольку агенты функционируют в разнообразных и иногда тяжело предсказуемых средах,
требуются надёжные методы тестирования, гарантирующие их корректную работу в условиях различных сбоев и аномалий.

Получается, тут напрашивается тот же подход хаос-инжиниринга, но смотреть нужно уже \grqqвглубь\glqq.

Хостовые агенты взаимодействуют с меньшим числом компонент, им не нужны широкие эксперименты, но нужны тонко настроенные,
учитывающие специфику окружения, вплоть до версии ОS (Operating system) и SO(Shared objects).
Использовать популярные инструменты хаос-тестирования для этого намного сложнее, к тому же возможность покрыть
специфичные сценарии вызывает вопросы.
Именно этим обеспечивается актуальность и новизна заявленной темы НИР.

\subsection{Чего боится хостовой агент?}

Следующим шагом в работе было определение ключевых требований к системе автоматического хаос-тестирования хостового агента.
Функциональные требования напрямую вытекали из критических точек отказа, о которых разработчики, как правило, не имеют возможности
побеспокоиться, оперируя классическими методами тестирования.
К таким точкам относятся оперативная память, CPU, диск, сеть, а также смежные относительно приложения утилиты,
с которыми агент плотно взаимодействует.
В каждой из точек потенциального сбоя были детально описаны губительные для агента процессы.

Сформулированные нефункциональные требования обусловлены принципами хаос-инжиниринга и потребностью в воспроизводимости
экспериментов.

\subsection{Что можно сделать уже сейчас?}

Следующим шагов в работе было оценить применимость и доступность существующих инструментов хаос-тестирования для хостового агента.
Были сделаны выводы о том, почему большинство инструментов не могут быть использованы для достижения целей.
Чаще всего причиной этому является их работа \grqqширокими мазками\glqq: они не могут отключить или сломать что-то маленькое,
их неделимая единица слишком велика для приложения, функционирующего на единственном поде.

Самым многообещающим показал себя сервис Chaos Mesh.
В его арсенале есть инструменты для внедрения сбоев в работу сети, процессора, оперативной памяти, операций ввода/вывода на диск.
Важным преимуществом сервиса является возможность проведения экспериментов, как и на виртуальных машинах, так и в окружении Kubernetes.
Именно с Chaos Mesh будут происходит дальнейшие манипуляции в НИР: ему предстоит проверки на возможность конфигурирования
специфичных сценариев и интеграции с дополнительными утилитами, в том числе для более детального взгляда на происходящее внутри
хостового приложения во время экспериментов.
Последнее является практически отступлением от одного из принципов хаос-инжиниринга: вместо метрик приложения,
акцент будет сделан на покрытии кода и веток.





