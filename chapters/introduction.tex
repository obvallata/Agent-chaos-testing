\newpage
\begin{center}
  \textbf{\large АННОТАЦИЯ}
\end{center}

Объектом исследования данной работы является система автоматизированного хаос-тестирования хостового агента.
Цель исследования — получение конфигурируемой среды, позволяющей оценивать поведение хостового агента
в условиях сбоев и делать выводы о доле покрытых сценариев работы агента.

К задачам исследования относятся:
\begin{enumerate}
  \item Описать требования к системе автоматизированного хаос-тестирования хостового агента.
  \item Изучить существующие инструменты хаос-тестирования и их применимость к хостовым агентам.
  \item Предложить решения для тех сценариев, которые остаются не покрытыми существующими решениями.
  \item Предложить способы детализации происходящего в хостовом приложении во время хаос тестирвоания.
  \item Провести серию тестов и проанализировать результаты.
\end{enumerate}

TODO: Полученные результаты и рекомендации, предложенные на основании данной работы.
+ Метрики качества + образ результата с уточнением про дополнительную информации о неожиданном месте деградации/отказа.

\onehalfspacing
\setcounter{page}{2}

\newpage
\renewcommand{\contentsname}{\centerline{\large СОДЕРЖАНИЕ}}
\tableofcontents

\newpage
\begin{center}
  \textbf{\large ВВЕДЕНИЕ}
\end{center}
\addcontentsline{toc}{chapter}{ВВЕДЕНИЕ}

\textbf{Актуальность}

В современном мире хостовые агенты становятся неотъемлемой частью информационных систем, обеспечивая мониторинг,
управление и безопасность на удалённых устройствах и серверах.
Они выполняют критически важные функции, такие как сбор данных, анализ производительности, обнаружение вторжений
и реагирование на инциденты.
Поскольку агенты функционируют в разнообразных и иногда тяжело предсказуемых средах,
требуются надёжные методы тестирования, гарантирующие их корректную работу в условиях различных сбоев и аномалий~\cite{Monperrus2017}.

Традиционные методы тестирования, такие как модульное (unit) тестирование и интеграционное тестирование,
часто не охватывают полный спектр возможных проблем, с которыми может столкнуться хостовый агент в реальной эксплуатации.
Модульное тестирование проверяет отдельные компоненты агента, часто не принимая во внимание взаимодействие с реальными внешними компонентами и окружением,
имитируя их функциональность с помощью моков.
Интеграционное тестирование проверяет взаимодействие между компонентами агента в том числе в окружении близком к продовому,
однако оно не расширяет множество этих сценариев проверки на неконсистентные и неожиданные условия среды.

При всем этом хостовой агент может столкнуться с различными проблемами в реальной эксплуатации, такими как:
\begin{enumerate}
  \item Сбои в сети: потеря соединения, задержка, потеря пакетов.
  \item Проблемы с памятью на хосте: высокая утилизация памяти, высокая утилизация процессора, read-only состояние диска, замедление операций на диске.
  \item Баги в исходном коде, которые не были видны на предыдущих этапах тестирования: ошибки в коде, уязвимости безопасности.
\end{enumerate}

Без дополнительного рассмотрения такие проблемы могут привести к сбоям в работе агента:
\begin{enumerate}
  \item Потери данных: особенно это актуально для агентов мониторинга и безопасности.
  \item Неконсистентное состояние агента с точки зрения функциональности: при неожиданном изменении конфигурации агент может выйти из привычного цикла работы.
  \item Неконсистентное состояние агента с точки зрения утечек ресурсов: из-за сбоев агент может неконтролируемо потреблять ресурсы хоста и негативно влиять на работу соседних сервисов.
\end{enumerate}

Для обеспечения надёжности и устойчивости хостовых агентов необходимо разработать методы тестирования,
которые будут учитывать возможные сбои и аномалии в их окружении.
Именно на этом сосредотачивается хаос-инжиниринг.
Хаос-инжиниринг — это подход к тестированию систем путём контролируемого внедрения сбоев.
Этот подход уже доказал свою эффективность в сфере серверных и облачных приложений~\cite{Basiri2016}.
Однако применение хаос-инжиниринга к хостовым агентам изучено недостаточно.

\newpage

\textbf{Цель выпускной квалификационной работы} -- получить систему автоматизированного хаос-тестирования
для хостового агента, которая позволит создавать контролируемые условия сбоев и ограничений в среде исполнения агента,
а также предоставить инструменты для мониторинга и анализа его поведения при возникновении этих сбоев.

\textbf{Задачи выпускной квалификационной работы:}

\begin{enumerate}
  \item Описать требования к системе автоматизированного хаос-тестирования хостового агента.
  \begin{itemize}
    \item Определить функциональные требования к системе.
    \item Определить нефункциональные требования к системе.
  \end{itemize}
  \item Исследовать существующие инструменты хаос-тестирования и их применимость к хостовым агентам.
    \begin{itemize}
      \item Проанализировать популярные инструменты и методы хаос-инжиниринга.
      \item Оценить возможности адаптации этих методов для хостовых агентов.
      \item Выбрать наиболее легко адаптируемый инструмент для хостового агента.
    \end{itemize}
  \item Описать сценарии тестирования, которые не затрагиваются существующими инструментами, и предложить для них варианты решений.
  \item Предложить способ мониторинга поведения хостового приложения во время хаос-тестирования.
  \item Провести серию тестов и проанализировать результаты.
    \begin{itemize}
      \item Запустить тестовые сценарии в выбранной среде.
      \item Оценить реакцию агента на различные сбои и собрать данные для анализ.
    \end{itemize}
\end{enumerate}
