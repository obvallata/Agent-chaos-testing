\newpage
\begin{center}
  \textbf{\large 1. ТРЕБОВАНИЯ К СИСТЕМЕ АВТОМАТИЗИРОВАННОГО ХАОС-ТЕСТИРОВАНИЯ ХОСТОВОГО АГЕНТА}
\end{center}
\refstepcounter{chapter}
\addcontentsline{toc}{chapter}{1. ТРЕБОВАНИЯ К СИСТЕМЕ АВТОМАТИЗИРОВАННОГО ХАОС-ТЕСТИРОВАНИЯ ХОСТОВОГО АГЕНТА}

\section{Что может пойти не так?}

При разработке и эксплуатации приложений на виртуальных машинах и железных хостах важно учитывать возможные проблемы,
которые могут возникнуть в результате работы инфраструктуры или самого приложения.
Эти проблемы могут затронуть различные ресурсы хоста, включая сеть, CPU, память и диск.
В условиях высокой утилизации ресурсов и неполадок хостовой агент может вести себя неожиданным образом,
и именно такие ситуации должны быть заранее изучены при помощи тестов.
Опишем подробнее варианты потенциальны сбоев на хосте~\cite{Blohowiak2016}.

\begin{enumerate}
  \item Проблемы с сетью:
  \begin{itemize}
    \item периодическая или полная потеря соединения с компонентами, к которым обращается агент;
    \item потеря пакетов;
    \item задержка в передаче пакетов.
  \end{itemize}
  \item Проблемы с CPU
  \begin{itemize}
    \item высокая утилизация CPU на хосте;
    \item низкие выставленные лимиты для хостового приложения.
  \end{itemize}
  \item Проблемы с оперативной памятью
  \begin{itemize}
    \item высокая утилизация памяти на хосте;
    \item низкие выставленные лимиты для хостового приложения.
  \end{itemize}
  \item \item Проблемы с диском
  \begin{itemize}
    \item высокая утилизация памяти на диске;
    \item низкие выставленные лимиты для хостового приложения;
    \item медленные I/O операции с диском;
    \item RO состояние диска.
  \end{itemize}
\end{enumerate}

Также могут возникать проблемы с отдельными утилитами и сервисами, с которыми взаимодействует конкретный хостовой агент.
Такие сбои будут специфичны для отдельного агента, но тоже должны конфигурироваться инструментом для тестирования~\cite{Simonsson2021}.

\section{Функциональные требования к системе}

Функциональные требования вытекают из описанных раннее составляющих сценариев хаос-тестов, специфичных для
хостового приложения.

\begin{enumerate}
  \item Высокая утилизация памяти
  \begin{enumerate}
    \item Система должна иметь возможность создавать видимость высокой нагрузки на оперативную память хоста,
    ограничивая доступную память для агента.
    \item Лимит доступной памяти должен конфигурироваться в зависимости от ожидаемого потребления памяти агентом.
  \end{enumerate}
  \item Высокая утилизация CPU
  \begin{enumerate}
    \item Система должна иметь возможность создавать видимость высокой нагрузки на CPU хоста,
    ограничивая доступность CPU для агента.
    \item Лимит доступности CPU (\%) должен конфигурироваться в зависимости от ожидаемого потребления.
  \end{enumerate}
  \item Система должна иметь возможность переводить диск в состояние RO.
  \item Замедление операций на диске.
  \begin{enumerate}
    \item Система должна иметь возможность увеличивать задержки операций ввода-вывода на диске.
    \item Пользователь должен иметь возможность настраивать величину задержки и пропускную способность диска.
  \end{enumerate}
  \item Сетевые сбои
  \begin{enumerate}
    \item Система должна иметь возможность создавать
    \begin{enumerate}
      \item сетевые задержки;
      \item потерю пакетов;
      \item ограничение пропускной способности сети;
      \item отключение сети.
    \end{enumerate}
    \item Пользователь должен иметь возможность настраивать параметры описанных выше сетевых воздействий.
  \end{enumerate}
  \item Мониторинг поведения хостового агента
  \begin{enumerate}
    \item Система должна собирать метрики производительности тестируемого приложения, включая использование CPU, памяти, диска, сети.
    \item Система должна предоставлять интерфейсы для визуализации и анализа собранных данных.
  \end{enumerate}
  \item Система должна предоставлять возможность имитировать проблемы в ответах внешних сервисов, к которым агент обращается.
  \item Система должна формировать отчет, который позволяют сделать вывод о степени покрытости множества сценариев исполнения хостовой программы.
\end{enumerate}

\section{Нефункциональные требования к системе}

Нефункциональные требования обусловлены необходимостью иметь возможность воспроизводить результаты тестов в тех же условиях,
а также одним из принципов хаос-тестирования~\cite{principles}, который говорит о том, что эксперименты не должны приносить вред внешним пользователям.

\begin{enumerate}
  \item Система должна управлять изолированным окружением, тестирование хостового агента в котором не будет
  ненамеренно затрагивать другие процессы.
  \item Система должна возвращать стабильные настройки окружения после окончания тестов.
  \item В случае непредвиденных сбоев система должна обеспечивать механизмы отката изменений.
\end{enumerate}